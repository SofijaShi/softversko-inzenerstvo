\documentclass[a4paper]{article}
\usepackage{amsmath}
\usepackage{amsfonts}
\usepackage{amssymb}
\usepackage{ucs}
\usepackage[T2A]{fontenc}
\usepackage[utf8]{inputenc}
\usepackage[english,bulgarian]{babel}
\usepackage{graphicx}
\usepackage{url}
\usepackage{textcomp}
\usepackage{tabularx}
\usepackage{array}
\usepackage[margin=1.5in]{geometry}
\usepackage[unicode]{hyperref}
\usepackage{listings}
\usepackage{color}
\definecolor{gray}{rgb}{0.6,0.6,0.6}
\usepackage[usenames,dvipsnames]{xcolor}
\lstset{language=C++,captionpos=b,
tabsize=4,frame=lines,
basicstyle=\script\ttfamily,
keywordstyle=\color{blue},
commentstyle=\color{gray},
stringstyle=\color{violet},
breaklines=true,showstringspaces=false}


\usepackage{fancyhdr}
\setlength{\headheight}{15pt}
 
\pagestyle{fancyplain}


\addto\captionsbulgarian{%
  \renewcommand{\contentsname}%
    {Содржина}%
  \renewcommand{\tablename}%
    {Табела}%
  \renewcommand{\figurename}%
    {Слика}%
  \renewcommand{\bibname}%
    {Библиографија}%
  \renewcommand{\listfigurename}%
    {Листа на слики}%
  \renewcommand{\listtablename}%
    {Листа на табели}%
}

\rhead{\textsc{Софтверско инженерство}}

\lhead{Аудиториски вежби 1}
\lfoot{}
\cfoot{\thepage}
\rfoot{}
\usepackage{fancyvrb}
\usepackage{xcolor}
\usepackage{textcomp}

\begin{document}

\section{Генеричко програмирање}

\subsection{Вовед}

Генеричкото програмирање претставува програмирање независно од типот на податоци
над кои ќе оперира. Тоа значи дека истата програма ќе работи со било кој тип на
податоци кои ќе и бидат проследени.

\begin{enumerate}
  \item Различно е имплементирано во различни јазици
  \item Се среќава во: Ada, BETA, C++, D, Eiffel, Java, C\#  
\end{enumerate}

Механизмот преку кој е реализирано генеричкото програмирање во C++ се шаблоните
(темплејти, templates).

Кај генеричките функции и класи типот на променливите со кои работат истите се
специфицира како параметар или се одредуваат од самиот преведувач, зависно од
нивната употреба. Ова значи дека функцијата или класата напишани како шаблон ќе
може да се искористат за повеќе различни типови на податоци без потребата да се
пишуваат различни верзии за секој тип на податоци.

Постојат два вида шаблони:
\begin{enumerate}
  \item Функциски темплејти (function templates)
  \item Класни темплејти (class templates)  
\end{enumerate}

Да ја претпоставиме следнава ситуација. Имаме функција која пресметува сума на
елементи на поле од цели броеви.

\lstinputlisting[firstline=1,lastline=6]{src/av1/av1_1.cpp}

Функцијата треба да се преоптеретува ако имаме поле на децимални броеви

\lstinputlisting[firstline=8,lastline=13]{src/av1/av1_1.cpp}

Во ваквите случаи се користат параметризираните функции за да се добие вистински
општа функција. Параметризираните функции можат да се сметаат како некој рецепт
или алгоритам за креирање на функција која може да биде употребена со општи
видови на податоци. Дефиницијата на параметризираните функции е многу слична на
дефиницијата на обичните функции со тоа што кај нив видовите на податоци што се
проследуваат, користат и враќаат од функциите се параметризирани. 

\emph{Кај параметризираните функции и класи видот на податоци со кои се оперира
се проследува како параметар.}

На овој начин се креира една функција или класа за
повеќе видови на податоци. Со параметризираните функции алгоритмите се кодираат
независно од податоците. 

\emph{Преведувачот е оној кој зависно од проследениот вид на податоци автоматски ја
генерира потребната функција.}

 На некој начин ова може да се нарече автоматско преоптоварување. Општата форма
 на овие функции е следнава
 
\begin{lstlisting}
template <class_ime>    
vid  ime_na funkcija( lista ) {
	// telo na funkcijata
}
\end{lstlisting}

\textsl{Пример 1}

\lstinputlisting{src/av1/ex1.cpp}

Клучните зборови \texttt{template} и \texttt{class} се употребуваат за дефинирање на
параметризираните функции. Тие покрај параметризирани функции (generic function)
се викаат и \textbf{шаблонски функции} (template function).

\subsection{Функциски темплејти}

Функциските темплејти се функции кои се параметризирани така да претставуваат
фамилија на функции. Овој вид на темплејти овозможуваат повикување на функциите
за различни податочни типови. Репрезентацијата на функцијата е слична на обична
функција, со таа разлика што некои елементи на функцијата се недефинирани. Овие
елементи се всушност параметризирани. 

Дефинирање на темплејт: Функциски темплејт кој го враќа максимумот од два
броеви.

\lstinputlisting[firstline=15,lastline=19]{src/av1/av1_1.cpp}

Типот на параметрите е оставен отворен како параметар темплејт \texttt{T}.
Темплејт параметрите мора да се наведат со следнава синтакса:

\begin{lstlisting}
template < comma-separated-list-of-parameters >
\end{lstlisting}

Клучниот збор \texttt{typename} означува таканаречен \texttt{type parameter}. Во
нашиот случај type parameter е \texttt{T}. Типот на овој параметар се наведува при повикот на
функцијата. Може да се користи било кој тип (основен податочен тип, класа итн.)
кој ја обезбедува операцијата која ја користи темплејтот. Од историски причини
може да се користи \texttt{class} наместо \texttt{typename} за да се дефинира
\texttt{type parameter}.

\textsl{Пример 2}

\lstinputlisting[lastline=8]{src/av1/ex2.cpp}

\subsection{Употреба на темплејти}

Употреба на \texttt{maximum()} функцискиот темплејт:

\lstinputlisting[firstline=10]{src/av1/ex2.cpp}

Програмата го дава следниов излез:

\begin{verbatim}
78
23.4
13334
\end{verbatim}

Првиот повик на функцијата:

\begin{lstlisting}
maximum (15 , 78)
\end{lstlisting}

го користи функцискиот темплејт со \texttt{int} како template parameter \texttt{T}.
\begin{lstlisting}
inline int const& maximum(int const& a, int const& b) {
	// if a < b then use b else use a
	return a < b ? b : a;
}
\end{lstlisting}

\emph{Процесот на замена на параметрите со конкретни податочни типови се нарекува
инстанцирање (instantiation).}

\begin{lstlisting}
const double& maximum(double const&, double const&);
const std::string& maximum(std::string const&, std::string const&);
\end{lstlisting}

Обид да се инстанцира темплејт за тип кој не ја подржува операцијата која се
користи резултира во грешка при компајлирање. 

\textsl{Пример}
\begin{lstlisting}
std::complex<float> c1, c2; // doesn't provide operator <
...
maximum(c1,c2); // ERROR at compile time
\end{lstlisting}


Темплејтите се компајлираат двапати:
\begin{enumerate}
  \item Без инстанцирање, темплејтите се проверуваат за коректна синтакса.
  \item За време на инстанцирањето, кодот се проверува за да се осигураме дека
сите повици се валидни.  
\end{enumerate}

Автоматска конверзија не е дозволена:
\begin{lstlisting}
template <typename T>
inline T const& maximum(T const& a, T const& b);
...
maximum(4, 7) // OK: T is int for both arguments
maximum(4, 4.2) // ERROR: first T is int, second T is double
\end{lstlisting}

Можни начини за справување со грешките:
\begin{enumerate}
  \item кастирање на аргументите:
  \begin{lstlisting}
  maximum(static_cast<double>(4),4.2) // OK
  \end{lstlisting}
  \item одредување на типот на T:
  \begin{lstlisting}
  maximum<double>(4, 4.2) // OK
  \end{lstlisting}
  \item наведување дека параметрите може да имаат различни типови. 
\end{enumerate}

\subsection{\texttt{Template} параметри и \texttt{call} параметри}

Функциските темплејти имаат два вида на параметри:
\begin{itemize}
  \item Template parameters, декларирани пред името на функцијата: 
  \begin{lstlisting}
  template <typename T> // T is template parameter
  \end{lstlisting}
  \item Call parameters,декларирани после името на функцијата: 
  \begin{lstlisting}
  maximum(T const& a, T const& b) // a and b are call parameters
  \end{lstlisting}
\end{itemize}

Дефинирање на функција со различен тип на call параметри:
\begin{lstlisting}
template <typename T1, typename T2>
inline T1 maximum(T1 const& a, T2 const& b) {
	return a < b ? b : a;
}
...
maximum(4, 4.2) // OK, but type of first argument defines return type
\end{lstlisting}

Можни проблеми:
\begin{itemize}
  \item мора да се наведе повратниот тип
  \item непожелно конвертирање од еден во друг повратен тип (пример: бараме
  максимум на 42 и 66.66 кој може да биде double 66.66 или int 66.
  \item конвертирањето на вториот аргумент за да одговара на повратниот тип
  креира нов локален објект заради што неможеме да го вратиме резултатот како
  референца (\texttt{T1} наместо \texttt{T1 const\&}).
\end{itemize}

Инстанцирање на темплејт за одреден податочен тип:

\begin{lstlisting}
template <typename T>
inline T const& maximum(T const& a, T const& b);
...
maximum<double>(4, 4.2) // instantiate T as double
\end{lstlisting}

употреба на трет аргумент за дефинирање на повратниот тип:

\begin{lstlisting}
template <typename T1, typename T2, typename RT>
inline RT maximum(T1 const& a, T2 const& b);
\end{lstlisting}

\textsl{Пример}
\begin{lstlisting}
template <typename T1, typename T2, typename RT>
inline RT maximum(T1 const& a, T2 const& b);
...
maximum<int, double, double>(4, 4.2) // OK, but tedious
\end{lstlisting}

процес на изведување на типовите на аргументите:
\begin{lstlisting}
template <typename RT, typename T1, typename T2>
inline RT maximum(T1 const& a, T2 const& b);
...
maximum<double>(4, 4.2) // OK: return type is double
\end{lstlisting}

\subsection{Преоптоварување на функциски темплејти}

Исто како обичните функции и функциските темплејти може да се преоптоваруваат.

\textsl{Пример}:

\lstinputlisting{src/av1/ex3.cpp}

преферирање на обичната функција:
\begin{lstlisting}
maximum(7, 42) // both int values match the nontemplate function perfectly
\end{lstlisting}

ако темплејтот обезбедува функција која подобро одговара на повикот се повикува
функцискиот темплејт:

\begin{lstlisting}
maximum(7.0, 42.0) // calls the maximum<double> (by argument deduction)
maximum('a', 'b'); // calls the maximum<char> (by argument deduction)
\end{lstlisting}

празна листа на аргументи за темплејтите ограничува повик само на темплејт
функциите:

\begin{lstlisting}
maximum<>(7, 42) // calls maximum<int> (by argument deduction)
\end{lstlisting}

Бидејќи автоматската конверзија не е присутна во темплејтите последниот повик е
всушност повик кон nontemplate функцијата (\texttt{'a'} и \texttt{42.7} се
конвертираат во \texttt{int}):

\begin{lstlisting}
maximum('a', 42.7) // only the nontemplate function allows different argument types
\end{lstlisting}

Следнава програма не работи еднакво за генерирани видови на класи

\lstinputlisting{src/av1/ex4.cpp}

\begin{verbatim}
maximum(10, 15) = 15
maximum('k', 's') = s
maximum(10.1, 15.2) = 15.2
maximum("Branko", "Aco") = Aco
\end{verbatim}

Зошто?

\subsection{Специјализирање на темплејт функција} 

Да се специјализира нова темплејт функција за да се корегира програмата

\lstinputlisting{src/av1/ex5.cpp}

\begin{verbatim}
maximum(10, 15) = 15
maximum('k', 's') = s
maximum(10.1, 15.2) = 15.2
maximum("Branko", "Aco") = Branko
\end{verbatim}


Генерирање на параметризирана функција baraj која ќе шребарува поле од објекти и
ќе го враќа индексот на најдениот објект или -1 ако не е најден.

\lstinputlisting{src/av1/ex6.cpp}


Генерирање параметризирана функција за сортирање на поле.

\lstinputlisting{src/av1/ex7.cpp}

\begin{verbatim}
 Sortiranje na double pole (golemina == 11) 
< 1.7  5.7  11.7  14.9  15.7  19.7  26.7  37.7  48.7  59.7  61.7   >
 Sortiranje na int pole (golemina == 16) 
< 61  87  154  170  275  426  503  509  512  612  653  677  703  765  897  908   >
\end{verbatim}

\subsection{Декларација на \texttt{Class Templates}}
\begin{lstlisting}
template <typename T>
class Stack {
	...
};
ili:
template <class T>
class Stack {
	...
};
\end{lstlisting}

Дадена е следнава едноставна класа

\begin{lstlisting}
#include <iostream>
class abc {
     char data;
public:
     abc(char x) { data = x; cout << data << endl; };
};
int main() {
	abc a('c');
	return 0;
}
\end{lstlisting}
Да се претвори во шаблонска класа и да се тестира на работа со цели и децимални
броеви.

\lstinputlisting{src/av1/ex8.cpp}

\subsection{Специјализирање на шаблонска класа}

Специјализираната шаблонска класа се дефинира на следниов начин

\begin{lstlisting}
template <> class class_name <type> 
\end{lstlisting}

Соодветно треба да се дефинираат и останатите функции членови.

Потребе е дури и посебен конструктор иако тој е ист со оној на
неспецијализираниот шаблон бидејќи специјализираниот шаблон не го наследува
основниот.
	
\lstinputlisting{src/av1/ex9.cpp}

\subsection{Примери и задачи}
Кон градба на шаблонска класа се приоѓа кога класата е можна за повеќе видови
податоци. Да претпоставиме дека сакаме да конструираме класа
\texttt{bazenVector} која ќе може да работи со хипотетичниот вид на податоци Т. Потребно е класата да
се изгради со следниве членови:

\begin{itemize}
  \item конструктор кој би креирал објекти од класата \texttt{bazenVector} со
  дадена големина
  \item деструктор
  \item препотоварен оператор \texttt{[]} за пристап до саканиот елемент
\end{itemize}

\lstinputlisting{src/av1/ex10.cpp}

\begin{verbatim}
dolz na A (15) e 15
dolz na B (5) e 5
podatoci vo A (0, 2, 4,...) se 0  2  4  6  8  10  12  14  16  18  20  22  24  26  28  
podatoci vo B (2.5, 5.5,...)se 2.5  5.5  8.5  11.5  14.5  
podatoci vo C ('f', 'g',...)se f  g  h  i  j  k  l  m
\end{verbatim}

Од оваа класа да се состави класа \texttt{iBazenVektor} за класа што ќе работи
само со цели броеви.

\lstinputlisting{src/av1/ex11.cpp}

\begin{verbatim}
dolz na A (15) e 15
podatoci vo A (0, 2, 4,...) se 0  2  4  6  8  10  12  14  16  18  20  22  24  26  28
dolz na A (15) e 15
\end{verbatim}

Слично може да се реализира и параметризирана класа за \texttt{Stack}.

\lstinputlisting{src/av1/ex12.cpp}

Програмата ќе испише
\begin{verbatim}
Pop s1: c
Pop s1: b
Pop s1: a
Pop s2: z
Pop s2: y
Pop s2: x
Pop ds1: 5.5
Pop ds1: 3.3
Pop ds1: 1.1
Pop ds2: 6.6
Pop ds2: 4.4
Pop ds2: 2.2
\end{verbatim}

Пример за параметризирана класа за \texttt{Queue}

\lstinputlisting{src/av1/ex13.cpp}

Програмата ќе испише
\begin{verbatim}
Zemi 1: 1
Zemi 2: A
Zemi 1: 2
Zemi 2: B
Zemi 1: 3
Zemi 2: C
Zemi 1: 4
Zemi 2: D
Zemi 1: 5
Zemi 2: E
Zemi 1: 6
Zemi 2: F
Zemi 1: 7
Zemi 2: G
Zemi 1: 8
Zemi 2: H
Zemi 1: 9
Zemi 2: I
Zemi 1: 10
Zemi 2: J
\end{verbatim}

\subsection{Шаблони и статички податочни членови}

Секоја класа или функција генерирана од шаблон има сопствени копии од статични
членови.
Во следнава пример програма

\begin{lstlisting}
template <class T>
class X {
    public:
		static T s;
};

int main() {
	X<int> xi;
 	X<char*> xc;
 	return 0;
}
\end{lstlisting}

\texttt{X<int>} има статичен член од видот \texttt{int} и \texttt{X<char*>} има
статичен член од видот \texttt{char*}.

Како се дефинираат статичните членови кај шаблонските класи ќе биде објаснето во
следнава пример програма:

\lstinputlisting{src/av1/ex14.cpp}


\begin{verbatim}
xi.s = 3
xc.s = Programiraj
\end{verbatim}

Друг пример

\lstinputlisting{src/av1/ex15.cpp}

\begin{verbatim}
s = 10
s = Nemoj da programiras
\end{verbatim}

\subsection{Шаблони и наследство}

\lstinputlisting{src/av1/ex16.cpp}

\begin{verbatim}
Klasa osnovna 94
Klasa nasledena 94
\end{verbatim}

\subsection{Парцијална специјализација}

\begin{lstlisting}
template <typename T1, typename T2>
class MyClass {
...
};
// partial specialization: both template parameters have same type
template <typename T>
class MyClass<T,T> {
...
};
// partial specialization: second type is int
template <typename T>
class MyClass<T,int> {
...
};
// partial specialization: both template parameters are pointer types
template <typename T1, typename T2>
class MyClass<T1*,T2*> {
...
};
MyClass<int,float> mif; // uses MyClass<T1,T2>
MyClass<float,float> mff; // uses MyClass<T,T>
MyClass<float,int> mfi; // uses MyClass<T,int>
MyClass<int*,float*> mp; // uses MyClass<T1*,T2*>

\end{lstlisting}

\end{document}
