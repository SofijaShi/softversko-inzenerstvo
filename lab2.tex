\documentclass[12pt,a4paper]{exam}
\usepackage{amsmath}
\usepackage{amsfonts}
\usepackage{amssymb}
\usepackage{ucs}
\usepackage[T2A]{fontenc}
\usepackage[utf8]{inputenc}
\usepackage[english,bulgarian]{babel}
\usepackage{listings}
\usepackage{color}
\definecolor{lightgrey}{rgb}{0.9,0.9,0.9}
\usepackage[usenames,dvipsnames]{xcolor}
\lstset{language=C,captionpos=b,
tabsize=4,frame=lines,
basicstyle=\tiny\ttfamily,
keywordstyle=\color{blue},
commentstyle=\color{lightgrey},
stringstyle=\color{violet},
breaklines=true,showstringspaces=false}


\begin{document}
\pagestyle{headandfoot}
\header{\textbf{ФИНКИ/ФЕИТ\\Софтверско
инженерство}}{}{\large{\textbf{Лабораториска вежба 2}}}
\headrule
\cfoot{Страна \thepage}
\begin{center}
\Large{\textbf{STL контејнери}}
\end{center}
\begin{questions}

\question
Напишете програма во која се внесува секвенца од бинарни броеви (цели броеви 1 и
0) од \texttt{cin} и се зачувува во контејнер. Ако се примат вредности различни
од 0 и 1 од \texttt{cin} програмата завршува со читањето. Откако ќе се сместат
вредностите во контејнерот, да се примени \texttt{``bit-stuffing''} алгоритмот
врз податоците во контејнерот. \texttt{Bit-stuffing} се користи при пренос на
податоци, имено после секоја појава на 5 последователни 1 се вметнува 0. Да се
испечати контејнерот во \texttt{cout}. Исто така да се испечати и апсолутното и
релативното проширување на бит секвенцата. Апсолутното проширување се мери во
битови, имено за колку битови пораснала секвенцата, додека релативното
проширување се мери во проценти. На модифицираната листа направете го обраниот
процес, исфрлете ги додадените 0 и споредете ја добиената секвенца со
оригиналната бит секвенца заради проверка.

\question
Да се напише класа која ќе ја опишува патната инфраструктура во една држава. Во класата се
чуваат информации за државата, градовите и патиштата. За државата се чува името,
а за секој град се чува неговото име, број на жители и листа на градови со кои е
поврзан со директен пат како и должината на тој пат.
Класата треба да ги имплементира следните методи:
\begin{itemize}
  \item \texttt{void search(string cityName)} - пребарува град и ги печати информациите
  за градот како името, бројот на жители и листата на градови со кој е поврзан. Доколку не
постои градот се фрла исклучок од типот \texttt{CityNotFoundException};
\item \texttt{float roadNetwork()} - ја враќа вкупната должина на сите директни патишта во
државата.
\item \texttt{void mostDense()} - ги печати имињата на двата градови чии што коефициент кој се
преметува со формулата \texttt{(жители на град1 + жители на град2) / растојание (помеѓу град1 и
град2)}, е најголем.  
\end{itemize}

\end{questions}
\end{document}